\chapter{期望}

\def\F{\mathcal{F}}
\def\s{$(\Omega,\F,\mu)$}
\def\R{\mathbb{R}}
\begin{definition}
    若\s{}为测度空间, $f$为非负简单函数, 即$f=\sum_{k=1}^{m}x_kI_{A_k}$, $x_k\geqslant 0$, $A_k\in\F$且两两不交, $\cup_{k=1}^{m}A_k=\Omega$, 则$f$在$\Omega$上对$\mu$的积分为
        \begin{equation}
            \int_{\Omega}f\,\d\mu=\sum_{k=1}^{m}x_k\mu(A_k).
        \end{equation}
\end{definition}

\begin{definition}
    若\s{}为测度空间, $f$为非负可测函数, 则$f$在$\Omega$上对$\mu$的积分为
        \begin{equation}
            \int_{\Omega}f\,\d\mu=\sup\left\{\int_{\Omega}g\,\d\mu\colon g\text{为非负简单函数}\right\}.
        \end{equation}
\end{definition}

\begin{definition}
    若\s{}为测度空间, $f$为可测函数, 则$f$在$\Omega$上对$\mu$的积分为
        \begin{equation}
            \int_{\Omega}f\,\d\mu=\int_{\Omega}f_+\,\d\mu-\int_{\Omega}f_-\,\d\mu,
        \end{equation}
    其中$f_+:=\max\left\{f,0\right\}$, $f_+:=-\min\left\{f,0\right\}$, 但若$\int_{\Omega}f_+\,\d\mu=+\infty$, $\int_{\Omega}f_-\,\d\mu=+\infty$, 则称$f$在$\Omega$上对$\mu$的积分不存在, 否则称$f$在$\Omega$上对$\mu$的积分存在, 并且若$\int_{\Omega}f\,\d\mu$有限, 则称$f$在$\Omega$上对$\mu$可积.
\end{definition}

\begin{definition}
    若\s{}为测度空间, $f$为复可测函数, $\int_{\Omega}\Re(f)\,\d\mu$, $\int_{\Omega}\Im(f)\,\d\mu$都存在, 则$f$在$\Omega$上对$\mu$的积分为
        \begin{equation}
            \int_{\Omega}f\,\d\mu=\int_{\Omega}\Re(f)\,\d\mu+i\int_{\Omega}\Im(f)\,\d\mu.
        \end{equation}
\end{definition}

\begin{definition}
    若\s{}为测度空间, $A\in\F$, $f$为复可测函数, 则$f$在$A$上对$\mu$的积分为
    \begin{equation}
        \int_{A}f\,\d\mu=\int_{\Omega}fI_A\,\d\mu.
    \end{equation}
\end{definition}

\begin{definition}
    若$(\mathbb{R}^n,{\mathscr{B}}^n,\mu_F)$为测度空间, $\mu_F$为分布函数$F$诱导出的Lebesgue-Stieltjes测度, $f$为复可测函数, 则$f$对$F$的Lebesgue-Stieltjes积分为
    \begin{equation}
        \int f\,\d F=\int f\,\d \mu_F.
    \end{equation}
\end{definition}

\begin{definition}
    若$(\Omega,\F,\mathbf{P})$为概率空间, $A\in\F$, $f$为复可测函数, 则称$f$在$A$上对$\mathbf{P}$的积分为$f$在$A$上对$\mathbf{P}$的期望, 记作$\mathbf{E}_{\mathbf{P}}[f;A]$.
\end{definition}

\begin{definition}
    若\s{}为测度空间, $A\in\Omega$, $\exists  B\in\F$, $B\supset A$且$\mu(B)=0$, 则称$A$为一$\mu$零集.
\end{definition}

\begin{definition}
    若\s{}为测度空间, 有一与$\omega\in\Omega$有关的性质$\mathscr{P}=\left\{\mathscr{P}(\omega)\colon \omega\in\Omega\right\}$, $\left\{\omega\in\Omega\colon \mathscr{P}(\omega)\text{不成立}\right\}$为$\mu$零集, 则称$\mathscr{P}$几乎处处成立, 简记作$\mathscr{P}$, a.e.. 若$\mu=\mathbf{P}$为概率, 则$\mathscr{P}$几乎处处成立又称$\mathscr{P}$几乎必然成立, 简记作$\mathscr{P}$, a.s..
\end{definition}

\begin{definition}
    若\s{}为测度空间, $F\in\F$, 对任意函数$f$, 若存在可测函数$g$, $f=g$, a.e., 则$f$在$F$上对$\mu$的积分为
    \begin{equation}
        \int_{F}f\,\d\mu=\int_{F}g\,\d\mu.
    \end{equation}
\end{definition}

\begin{definition}
    若$(\Omega,\F)$和$(\Omega',\F')$为测度空间, 函数$f\colon \Omega\to\Omega'$满足$\forall A' \in \F'$, $f^{-1}(A')\in\F$ (可测集的原像是可测集), 则称$f$为$(\Omega,\F)$到$(\Omega',\F')$的可测映射.
\end{definition}

\begin{definition}
    若$(\Omega,\F)$和$(\Omega',\F')$为测度空间, $\mu$为$\F$上的测度, $f$为$(\Omega,\F)$到$(\Omega',\F')$的可测映射, $\mu_f\colon \F'\to \bar{\mathbb{R}}_+,A'\mapsto\mu_f(A'):=\mu(f^{-1}(A'))$, 则$\mu_f$为测度, 称为$\mu$在$(\Omega',\F')$上由$f$导出的测度.
\end{definition}

\begin{theorem}
    (积分变换定理)若$(\Omega,\F)$和$(\Omega',\F')$为测度空间, $\mu$为$\F$上的测度, $f$为$(\Omega,\F)$到$(\Omega',\F')$的可测映射, $g$为$(\Omega',\F')$上的可测函数, 则
    \begin{equation}
        \int_{f^{-1}(A')}(g\circ f)\,\d\mu=\int_{A'}g\,\d\mu_f,
    \end{equation}
    上式的意义是: 若等式的一边存在, 则另一边存在, 且二者相等.
\end{theorem}

\begin{definition}
    若\s{}为测度空间, 对任意函数$f$, 若$f$的积分存在, 则$f$在$\F$上对$\mu$的不定积分为$\int f\,\d\mu\colon \F\to\bar{\mathbb{C}},A\mapsto\left[\int f\,\d\mu\right](A)$, 其中
    \begin{equation}
        \left[\int f\,\d\mu\right](A)=\int_A f\,\d\mu.
    \end{equation}
\end{definition}
